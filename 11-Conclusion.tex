\chapter*{Conclusion générale et intérêts}

\label{sec:conclusion}

Pour conclure ce mémoire, notre exploration approfondie des plateformes de co-simula-\\tion et des algorithmes d'orchestration a non seulement démontré leur potentiel transformateur mais aussi leur valeur ajoutée substantielle pour l'équipe RT-sim. L'adoption d'OMSimulator, offre à l'équipe une plateforme de co-simulation plus robuste, facilitant ainsi l'intégration et les tests de nouvelles approches. Cette amélioration marque un tournant, permettant à l'équipe RT-sim de gagner un temps précieux lors de l'incorporation et des essais des innovations.

L'introduction d'une taxonomie basée sur des principes ontologiques s'avère particulièrement bénéfique pour les nouveaux arrivants au projet, car elle facilite la segmentation et la compréhension du sujet.Cette structure bien définie facilite la compréhension des interactions et des rôles des différentes entités impliquées, simplifiant ainsi la recherche bibliographique pour les nouveaux membres du projet.

En outre, nous avons mis en lumière la valeur substantielle des plateformes de co-simulation et des algorithmes d'orchestration, soulignant particulièrement leur rôle crucial dans la gestion efficace et précise de simulations complexes. À travers une analyse comparative des plateformes open source disponibles, OMSimulator s'est distingué comme étant particulièrement adapté à nos besoins grâce à sa robustesse, et sa documentation exhaustive. La mise en œuvre de l'algorithme d'orchestration Aitken-Schwarz a prouvé son efficacité, améliorant significativement la gestion des erreurs et la stabilité des simulations, validant ainsi l'efficacité de notre approche choisie. Cependant, il a été observé que dans certains scénarios, la méthode Aitken-Schwarz n'a pas réussi à produire de bons résultats. Cela met en évidence l'intérêt pour des méthodes prédictives telles que la méthode COSTARICA \cite{31}, qui utilise des techniques avancées pour estimer et corriger les erreurs en temps réel. 

Les progrès réalisés dans le domaine de la co-simulation peuvent positionner cette technologie comme une alternative supérieure à la simulation monolithique, notamment pour les applications qui nécessitent une collaboration interdisciplinaire. Cela souligne l'importance et l'impact des recherches menées par l'équipe RT-SIM de Capgemini, qui œuvre à optimiser et à étendre les capacités de la co-simulation. En intégrant des avancées telles que les méthodes Aitken-Schwarz, ces recherches contribuent à transformer la co-simulation en un outil plus flexible, précis et efficace, adapté aux défis complexes des systèmes modernes. Cette orientation vers des solutions innovantes illustre l'engagement de l'équipe à fournir des solutions de pointe qui améliorent non seulement la performance mais aussi l'intégrabilité des différents systèmes dans des contextes multidisciplinaires.