\mychapter{0}{Résumé}

La recherche de nouvelles approches de collaboration interdisciplinaire est essentielle pour naviguer dans les complexités du développement de systèmes face à des exigences de marché en augmentation. Une approche envisageable pour surmonter ce défi est l'adoption d'une méthode basée sur des modèles hétérogènes. Cette méthode permet à diverses équipes de créer leurs propres modèles et de conduire leurs analyses habituelles dans leur discipline respective. De plus, elle offre la possibilité de coupler ces modèles pour réaliser des simulations conjointes (co-simulation), facilitant ainsi l'analyse du comportement global du système. La norme Functional Mock-up Interface (FMI), développée par l'association Modelica, joue un rôle crucial dans la facilitation de cette intégration. FMI fournit un cadre polyvalent pour l'échange aisé de modèles et de données de simulation entre différents outils et plates-formes.
\medskip

Dans ce travail, nous présentons une approche globale de la taxonomie de la co-simulation, en utilisant des notions ontologiques pour segmenter en définissant systématiquement les divers éléments et interactions au sein des environnements de co-simulation. Cette taxonomie structurée sert de base pour améliorer la clarté et l'interopérabilité des processus de co-simulation. En outre, nous nous concentrons sur l'optimisation de l'orchestrateur d'OMSimulator, une plateforme conçue pour la co-simulation basée sur le standard FMI. Pour ce faire, nous avons employé une méthode itérative appelée méthode d'Aitken-Schwarz. En intégrant la méthode d'Aitken-Schwarz dans l'orchestrateur d'OMSimulator, nous avons pu améliorer significativement la stabilité des simulations numériques, réduire les erreurs et assurer une convergence plus rapide des résultats.
\medskip

Ces améliorations permettent de renforcer les processus de co-simulation, offrant ainsi une solution plus robuste et efficace pour le développement de systèmes complexes dans un environnement interdisciplinaire.

\medskip




\vspace{1cm}


\noindent\rule[2pt]{\textwidth}{0.5pt}

{\textbf{Mots clés :}}
Co-simulation, FMI, Taxonomie, OMSimulator, Aitken-Schwarz.
\\
\noindent\rule[2pt]{\textwidth}{0.5pt}

\clearpage

\mychapter{0}{Abstract}


The search for new approaches to interdisciplinary collaboration is essential for navigating the complexities of system development in the face of increasing market demands. One conceivable approach to overcome this challenge is the adoption of a method based on heterogeneous models. This method allows various teams to create their standard models and conduct their usual analyses within their respective disciplines. Additionally, it offers the possibility of coupling these models to perform joint simulations (co-simulation), thus facilitating the analysis of the overall system behavior. The Functional Mock-up Interface (FMI) standard, developed by the Modelica Association, plays a crucial role in facilitating this integration. FMI provides a versatile framework for the easy exchange of models and simulation data between different tools and platforms.
\medskip

In this work, we present a comprehensive approach to the taxonomy of co-simulation, using ontological notions to systematically categorize and define the various elements and interactions within co-simulation environments. This structured taxonomy serves as a basis for improving the clarity and interoperability of co-simulation processes. Furthermore, we focus on optimizing the orchestrator of OMSimulator, a platform designed for co-simulation based on the FMI standard. To achieve this, we employed an iterative method called the Aitken-Schwarz method. By integrating the Aitken-Schwarz method into the OMSimulator orchestrator, we were able to significantly improve the stability of numerical simulations, reducing errors and ensuring faster convergence of results.
\medskip

These improvements enhance the co-simulation processes, thus providing a more robust and efficient solution for the development of complex systems in an interdisciplinary environment.
\medskip

\vspace{1cm}



\noindent\rule[2pt]{\textwidth}{0.5pt}

{\textbf{Keywords :}}
Co-simulation, FMI, Taxonomy, OMSimulator, Aitken-Schwarz.
\\
\noindent\rule[2pt]{\textwidth}{0.5pt}


