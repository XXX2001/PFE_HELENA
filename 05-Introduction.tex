\selectlanguage{french}
\chapter*{Introduction générale}
\addcontentsline{toc}{chapter}{Introduction générale}
\markboth{Introduction générale}{Introduction générale}
\label{chap:introduction}
%\minitoc

\section*{Contexte}

Dans le domaine technologique, qui évolue rapidement, le développement de systèmes complexes exige souvent une collaboration entre plusieurs disciplines. Que ce soit en ingénierie aérospatiale, en conception automobile ou dans d'autres domaines, il est essentiel d'adopter des méthodes avancées pour naviguer à travers les complexités et interdépendances croissantes des composants de ces systèmes. Les approches traditionnelles, qui fonctionnent souvent dans des disciplines séparées, s'avèrent insuffisantes pour répondre aux exigences croissantes en matière d'efficacité, d'innovation et de performance. Il existe donc un besoin critique de nouvelles méthodologies collaboratives capables d'intégrer diverses expertises et de faciliter l'analyse complète des systèmes. L'utilisation de modèles hétérogènes et de techniques de co-simulation constitue une approche prometteuse pour relever ce défi. En permettant à différentes équipes de créer et d'analyser leurs modèles indépendamment tout en permettant des simulations intégrées, ces techniques offrent une voie vers un développement des systèmes plus robustes et plus précis. Le standard Functional Mock-up Interface (FMI), développé par l'association Modelica, apparaît comme un cadre essentiel dans le contexte de cette étude, fournissant une plateforme unifiée pour l'échange de modèles et de données entre différents outils de simulation.

\section*{Problématique}

La co-simulation, outil crucial pour analyser et développer des systèmes complexes, est souvent abordée de façon spécifique par les différents communautés scientifiques et industrielles. Cette approche fragmentée mène à un manque de standardisation et de collaboration, rendant difficile l'établissement d'une base commune et solide pour le développement du domaine. Par conséquent, les équipes interdisciplinaires rencontrent des difficultés pour adopter cette méthode, ce qui limite son utilisation et freine l'innovation dans le secteur.

En outre, la majorité des plateformes de co-simulation existantes se trouvent soit à un stade de développement alpha, soit sont insuffisamment documentées, utilisant souvent des langages propriétaires. Cette situation pose de nombreuses contraintes à l'intégration et à l'incorporation d'algorithmes d'orchestration avancés. Les utilisateurs se retrouvent face à des défis significatifs pour implémenter et optimiser ces algorithmes dans des environnements de co-simulation non standardisés, ce qui limite l'efficacité et la fiabilité des simulations numériques. Ainsi, il est impératif de développer des solutions robustes et bien documentées, basées sur des normes ouvertes comme le standard (FMI), afin de surmonter ces obstacles et de faciliter une adoption plus large et plus efficace des méthodes de co-simulation dans divers domaines.

\section*{Objectifs}
Le premier objectif de cette étude est de développer une taxonomie simplifiée et basée sur des diagrammes pour la co-simulation. En utilisant une structure ontologique claire, nous allons catégoriser et définir systématiquement les différents éléments et interactions dans les environnements de co-simulation. Les diagrammes explicatifs rendront cette approche plus accessible et intuitive pour des équipes interdisciplinaires.

Le deuxième objectif est d'étudier les différentes plateformes de co-simulation existantes. Nous analyserons les caractéristiques, avantages et limitations des principales plateformes, en mettant en lumière les défis liés à l'utilisation de langages propriétaires. Cette étude permettra d'identifier les meilleures pratiques et les lacunes actuelles.

Enfin, le troisième objectif est de mettre en place un algorithme itératif d'orchestration pour la co-simulation. Cet algorithme sera utilisé pour renforcer la stabilité des simulations numériques, minimiser les erreurs et permettre une convergence accélérée des résultats.
\section*{Organisation du mémoire}

Ce rapport est organisé en trois chapitres :

Le premier chapitre \og \textbf{\hyperref[sec:pré]{Présentation de l'organisme d'accueil et cadrage du projet}} \fg présente l'entreprise d'accueil, la méthodologie, le planning, les contraintes liées au sujet, ainsi que le cadrage du projet.

Le deuxième chapitre \og \textbf{\hyperref[sec:tax]{Taxonomie de la co-simulation}
} \fg présente une catégorisation systématique et une définition des différents éléments et interactions au sein des environnements de co-simulation en utilisant des notions ontologiques.

Le troisième chapitre \og \textbf{\hyperref[sec:orc]{Conception et développement d'un orchestrateur de co-simulation}
} \fg  explore une comparaison détaillée des différentes plateformes de co-simulation disponibles, mettant en lumière leurs caractéristiques, avantages et inconvénients. De plus, ce chapitre examine l'amélioration de l'algorithme d'orchestration, interprète les résultats obtenus et les compare avec ceux issus d'autres outils de co-simulation existants.
\medskip


\clearpage